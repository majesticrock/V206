\section{Diskussion}
\label{sec:Diskussion}
Da die thermischen Isolierungen nicht perfekt seien können, verändert sich das System jederzeit.
Bedingt dadurch, dass die Messwerte nicht alle zeitgleich aufgenommen werden können, sorgt dieser Effekt zusätzlich zu der normal bereits vorhandenen Zeitabweichung für einen erhöhten Fehler.
Bis auf die Thermometer sind alle Messgeräte mit einer analogen Skala versehen, welche nicht exakt abgelesen werden kann. Explizit bei der Messung von $p_b$ fing der Zeiger nach ungefähr der halben Messreihe an zu zucken.
Auch die Wassermenge kann nur mit menschlicher Genauigkeit genommen werden.