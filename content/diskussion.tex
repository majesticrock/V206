\section{Diskussion}
\label{sec:Diskussion}
<<<<<<< HEAD
Die Ergebnisse weichen teilweise stark von ihrem Idealwert ab. Während der Massendurchsatz von der Größeordnung her passt, weicht die Güteziffer stark vom bestimmten Idealwert ab.
Auch die bestimmte Kompressorleistung ist wesentlich geringer, als die tatsächlich investierte Leistung.
Diese Ergebnisse waren allerdings abzusehen, da die gesamte Messreihe aus verschieden Quellen her, stark fehlerbehaftet ist.
Zum einen können die thermischen Isolierungen nicht perfekt sein, daher verändert sich das System jederzeit.
||||||| merged common ancestors
Da die thermischen Isolierungen nicht perfekt seien können, verändert sich das System jederzeit.
=======
%TODO: Herkunft der Fehler erläutern
%TODO: Wie stark ist die Abweichung von dem Idealwert?
Da die thermischen Isolierungen nicht perfekt seien können, verändert sich das System jederzeit.
>>>>>>> verbesserung I
Bedingt dadurch, dass die Messwerte nicht alle zeitgleich aufgenommen werden können, sorgt dieser Effekt zusätzlich zu der normal bereits vorhandenen Zeitabweichung für einen erhöhten Fehler.
Außerdem sind bis auf die Thermometer alle Messgeräte mit einer analogen Skala versehen, welche nicht exakt abgelesen werden kann. Explizit bei der Messung von $p_b$ fing der Zeiger nach ungefähr der halben Messreihe an zu schwingen.
Auch die Wassermenge kann nur mit menschlicher Genauigkeit genommen werden.
Erschwerend hinzu kommt, dass die Fehlerabschätzungen mit denen hier gearbeitet wurden, ebenfalls nur geschätzt sind und nicht dem tatsächlichen Fehler der Messgeräte entsprechen.
Alles in allem liegen die hier vorliegend Ergebnisse zwar in der Größenordnung der Erwartungen, jedoch ist aus den hier genannten Gründen nicht davon auszugehen, dass bei einer erneuten Durchführung reproduzierbare Ergebnisse erzielt werden würden.
Für bessere Ergebnisse könnte eine häufigere Durchführung sorgen, über deren Ergebnisse gemittelt werden müsste.
Des Weiteren wären einerseits genauere Messgeräte, insbesondere solche die alle Werte gleichzeitig aufnehmen könnten, sowie eine verbesserte Wärmeisolierung, 
wobei angemerkt werden muss, dass auch solche Maßnahmen, bei diesem Versuch niemals zu sehr viel besseren Ergebnissen führen würden.
